%% ================================================================================
%% This LaTeX file was created by AbiWord.                                         
%% AbiWord is a free, Open Source word processor.                                  
%% More information about AbiWord is available at http://www.abisource.com/        
%% ================================================================================

\documentclass[a4paper,portrait,12pt]{article}
\usepackage[latin1]{inputenc}
\usepackage{calc}
\usepackage{setspace}
\usepackage{fixltx2e}
\usepackage{graphicx}
\usepackage{multicol}
\usepackage[normalem]{ulem}
%% Please revise the following command, if your babel
%% package does not support fr-FR
\usepackage[frenchb]{babel}
\usepackage{color}
\usepackage{hyperref}
 
\begin{document}


\begin{flushleft}
Ingénierie des connaissances - HMIN231
\end{flushleft}


\begin{flushleft}
HMIN231 - SESSION DISTANCIEL MAI 2020 - Entre le 21 avril et le 5 mai
\end{flushleft}





\begin{flushleft}
Ce document contient les éléments que vous devez avoir à portée de la main pour réaliser le
\end{flushleft}


\begin{flushleft}
test en ligne. Prenez-en connaissance avant le démarrage du test (1 tentative).
\end{flushleft}





\begin{flushleft}
La figure 1 présente des données sur quelques représentations de plats (Dish), décrits par leur relation (has
\end{flushleft}


\begin{flushleft}
main cereal (hmc)) avec la céréale (Cereal ) qui y est principalement utilisée (ici des variétés de riz). Les variétés
\end{flushleft}


\begin{flushleft}
de céréales sont à leur tour décrites par leur relation (is produced in (ipi)) avec le pays (Country), où elles sont
\end{flushleft}


\begin{flushleft}
produites, tandis qu'une troisième relation indique quels plats mangent en quantité les habitants d'un pays (eat
\end{flushleft}


\begin{flushleft}
lot of (elo)) 1 . Les mêmes données (augmentées par des identifiants pour les objets) sont également présentées
\end{flushleft}


\begin{flushleft}
dans la table 1.
\end{flushleft}





\begin{flushleft}
Figure 1 -- Objets, leurs attributs et leurs liens
\end{flushleft}





\begin{flushleft}
La figure 2 présente les treillis de concepts construits par le processus itératif de l'Analyse Relationnelle de
\end{flushleft}


\begin{flushleft}
Concepts (étape finale) en utilisant le quantifieur $\exists$ sur toutes les relations et à toutes les étapes.
\end{flushleft}





\begin{flushleft}
1. L'exemple est tiré de : A Guided Tour of Artificial Intelligence Research, Pierre Marquis, Odile Papini, Henri Prade (editors),
\end{flushleft}


\begin{flushleft}
volume II, chapter 13, Formal Concept Analysis : From Knowledge Discovery to Knowledge,Processing, Sébastien Ferré, Marianne
\end{flushleft}


\begin{flushleft}
Huchard, Mehdi Kaytoue, Sergei O. Kuznetsov and Amedeo Napoli (authors), Springer International Publishing, to appear.
\end{flushleft}





\begin{flushleft}
HMIN231 - SESSION DISTANCIEL MAI 2020
\end{flushleft}





\begin{flushleft}
Entre le 21 avril et le 5 mai
\end{flushleft}





\begin{flushleft}
\newpage
Ingénierie des connaissances - HMIN231
\end{flushleft}





×


×





2





×


×


×





×


×


×





×


×


\begin{flushleft}
khaoManKai
\end{flushleft}





\begin{flushleft}
gardiane
\end{flushleft}





×





\begin{flushleft}
biryani
\end{flushleft}





×





\begin{flushleft}
eatLotOf (elo)
\end{flushleft}


\begin{flushleft}
Italy
\end{flushleft}


\begin{flushleft}
France
\end{flushleft}


\begin{flushleft}
Thailand
\end{flushleft}


\begin{flushleft}
Pakistan
\end{flushleft}





\begin{flushleft}
Asia
\end{flushleft}





\begin{flushleft}
Europe
\end{flushleft}





\begin{flushleft}
Pakistan
\end{flushleft}





×





\begin{flushleft}
arancini
\end{flushleft}





×





\begin{flushleft}
Thailand
\end{flushleft}





\begin{flushleft}
France
\end{flushleft}





\begin{flushleft}
Italy
\end{flushleft}





\begin{flushleft}
wheat
\end{flushleft}





\begin{flushleft}
rice
\end{flushleft}





\begin{flushleft}
country
\end{flushleft}


\begin{flushleft}
Italy
\end{flushleft}


\begin{flushleft}
France
\end{flushleft}


\begin{flushleft}
Thailand
\end{flushleft}


\begin{flushleft}
Pakistan
\end{flushleft}


\begin{flushleft}
Thailand
\end{flushleft}





\begin{flushleft}
Pakistan
\end{flushleft}





×





\begin{flushleft}
Italy
\end{flushleft}





\begin{flushleft}
thaiRice
\end{flushleft}





\begin{flushleft}
basmatiRice
\end{flushleft}





\begin{flushleft}
arborioRice
\end{flushleft}


×





\begin{flushleft}
isProducedIn (ipi)
\end{flushleft}


\begin{flushleft}
redRice
\end{flushleft}


\begin{flushleft}
arborioRice
\end{flushleft}


\begin{flushleft}
basmatiRice
\end{flushleft}


\begin{flushleft}
thaiRice
\end{flushleft}





\begin{flushleft}
HMIN231 - SESSION DISTANCIEL MAI 2020
\end{flushleft}





×





×


×


×


×





×





\begin{flushleft}
France
\end{flushleft}





×





×


×





\begin{flushleft}
basmatiRice
\end{flushleft}





×





\begin{flushleft}
thaiRice
\end{flushleft}





×


\begin{flushleft}
arborioRice
\end{flushleft}





\begin{flushleft}
hasMainCereal (hmc)
\end{flushleft}


\begin{flushleft}
arancini
\end{flushleft}


\begin{flushleft}
gardiane
\end{flushleft}


\begin{flushleft}
khaoManKai
\end{flushleft}


\begin{flushleft}
biryani
\end{flushleft}





×





\begin{flushleft}
redRice
\end{flushleft}





\begin{flushleft}
biryani
\end{flushleft}





\begin{flushleft}
khaoManKai
\end{flushleft}





\begin{flushleft}
gardiane
\end{flushleft}





\begin{flushleft}
cereal
\end{flushleft}


\begin{flushleft}
redRice
\end{flushleft}


\begin{flushleft}
arborioRice
\end{flushleft}


\begin{flushleft}
basmatiRice
\end{flushleft}


\begin{flushleft}
thaiRice
\end{flushleft}





×





\begin{flushleft}
redRice
\end{flushleft}





\begin{flushleft}
dish
\end{flushleft}


\begin{flushleft}
arancini
\end{flushleft}


\begin{flushleft}
gardiane
\end{flushleft}


\begin{flushleft}
khaoManKai
\end{flushleft}


\begin{flushleft}
biryani
\end{flushleft}





\begin{flushleft}
arancini
\end{flushleft}





\begin{flushleft}
Table 1 -- Relational Context Family : Formal contexts dishes, cereal, countries, and relations hasMainCereal,
\end{flushleft}


\begin{flushleft}
isProducedIn, eatLotOf. Dans les contextes formels, les 4 premiers attributs sont des identifiants.
\end{flushleft}





×


×





\begin{flushleft}
Entre le 21 avril et le 5 mai
\end{flushleft}





\begin{flushleft}
\newpage
Ingénierie des connaissances - HMIN231
\end{flushleft}





\begin{flushleft}
dish5
\end{flushleft}


\begin{flushleft}
ex. hmc(cereal5)
\end{flushleft}





\begin{flushleft}
dish6
\end{flushleft}


\begin{flushleft}
dish7
\end{flushleft}


\begin{flushleft}
ex. hmc(cereal3)
\end{flushleft}


\begin{flushleft}
ex. hmc(cereal6)
\end{flushleft}


\begin{flushleft}
ex. hmc(cereal7)
\end{flushleft}





\begin{flushleft}
dish4
\end{flushleft}


\begin{flushleft}
arancini
\end{flushleft}


\begin{flushleft}
arancini
\end{flushleft}





\begin{flushleft}
dish3
\end{flushleft}





\begin{flushleft}
cereal5
\end{flushleft}


\begin{flushleft}
rice
\end{flushleft}


\begin{flushleft}
ex. ipi(country7)
\end{flushleft}





\begin{flushleft}
dish2
\end{flushleft}





\begin{flushleft}
dish1
\end{flushleft}





\begin{flushleft}
gardiane
\end{flushleft}


\begin{flushleft}
khaoManKai
\end{flushleft}


\begin{flushleft}
biryani
\end{flushleft}


\begin{flushleft}
ex. hmc(cereal4) ex. hmc(cereal1) ex. hmc(cereal2)
\end{flushleft}


\begin{flushleft}
gardiane
\end{flushleft}





\begin{flushleft}
khaoManKai
\end{flushleft}





\begin{flushleft}
dish0
\end{flushleft}





\begin{flushleft}
biryani
\end{flushleft}





\begin{flushleft}
country7
\end{flushleft}





\begin{flushleft}
ex. hmc(cereal0)
\end{flushleft}





\begin{flushleft}
cereal7
\end{flushleft}





\begin{flushleft}
cereal6
\end{flushleft}





\begin{flushleft}
ex. ipi(country6) ex. ipi(country5)
\end{flushleft}





\begin{flushleft}
cereal4
\end{flushleft}





\begin{flushleft}
cereal3
\end{flushleft}





\begin{flushleft}
cereal2
\end{flushleft}





\begin{flushleft}
cereal1
\end{flushleft}





\begin{flushleft}
redRice
\end{flushleft}


\begin{flushleft}
arborioRice
\end{flushleft}


\begin{flushleft}
basmatiRice
\end{flushleft}


\begin{flushleft}
thaiRice
\end{flushleft}


\begin{flushleft}
ex. ipi(country3) ex. ipi(country4) ex. ipi(country1) ex. ipi(country2)
\end{flushleft}


\begin{flushleft}
redRice
\end{flushleft}





\begin{flushleft}
country6
\end{flushleft}


\begin{flushleft}
Europe
\end{flushleft}





\begin{flushleft}
country4
\end{flushleft}





\begin{flushleft}
country3
\end{flushleft}





\begin{flushleft}
Italy
\end{flushleft}





\begin{flushleft}
France
\end{flushleft}





\begin{flushleft}
Italy
\end{flushleft}





\begin{flushleft}
France
\end{flushleft}





\begin{flushleft}
arborioRice
\end{flushleft}





\begin{flushleft}
basmatiRice
\end{flushleft}





\begin{flushleft}
thaiRice
\end{flushleft}





\begin{flushleft}
country5
\end{flushleft}


\begin{flushleft}
Asia
\end{flushleft}


\begin{flushleft}
ex. elo(dish5)
\end{flushleft}


\begin{flushleft}
ex. elo(dish7)
\end{flushleft}





\begin{flushleft}
cereal0
\end{flushleft}


\begin{flushleft}
wheat
\end{flushleft}


\begin{flushleft}
ex. ipi(country0)
\end{flushleft}





\begin{flushleft}
country2
\end{flushleft}





\begin{flushleft}
country1
\end{flushleft}





\begin{flushleft}
Thailand
\end{flushleft}


\begin{flushleft}
ex. elo(dish2)
\end{flushleft}





\begin{flushleft}
Pakistan
\end{flushleft}


\begin{flushleft}
ex. elo(dish1)
\end{flushleft}





\begin{flushleft}
Thailand
\end{flushleft}





\begin{flushleft}
Pakistan
\end{flushleft}





\begin{flushleft}
country0
\end{flushleft}


\begin{flushleft}
ex.
\end{flushleft}


\begin{flushleft}
ex.
\end{flushleft}


\begin{flushleft}
ex.
\end{flushleft}


\begin{flushleft}
ex.
\end{flushleft}





\begin{flushleft}
elo(dish4)
\end{flushleft}


\begin{flushleft}
elo(dish0)
\end{flushleft}


\begin{flushleft}
elo(dish3)
\end{flushleft}


\begin{flushleft}
elo(dish6)
\end{flushleft}





\begin{flushleft}
Figure 2 -- Treillis de concepts sur les plats. ex.r(C) correspond à $\exists$r(C)
\end{flushleft}





\begin{flushleft}
HMIN231 - SESSION DISTANCIEL MAI 2020
\end{flushleft}





3





\begin{flushleft}
Entre le 21 avril et le 5 mai
\end{flushleft}





\newpage



\end{document}
